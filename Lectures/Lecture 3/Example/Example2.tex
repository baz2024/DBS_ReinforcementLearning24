\documentclass{article}
\usepackage{amsmath}

\begin{document}
 
\begin{Large}
Example 2: The Bellman equation of the Q function
\end{Large}
\\
Consider a simple grid world where an agent can move left, right, up, or down. The grid has a reward of $-1$ for each step, and the agent receives a reward of $+10$ for reaching the goal state. The discount factor $\gamma$ is set to $0.9$.

The Bellman equation for the Q-function $Q(s, a)$ of a state-action pair $(s, a)$ in this grid world is:

\[
Q(s, a) = R(s, a) + \gamma \sum_{s'} P(s' | s, a) \max_{a'} Q(s', a')
\]

where:
\begin{itemize}
    \item $s$ is the current state,
    \item $a$ is the action taken,
    \item $R(s, a)$ is the immediate reward for taking action $a$ in state $s$,
    \item $\gamma$ is the discount factor,
    \item $P(s' | s, a)$ is the probability of transitioning to state $s'$ from state $s$ after taking action $a$,
    \item $\max_{a'} Q(s', a')$ is the maximum Q-value for the next state $s'$ over all possible actions $a'$.
\end{itemize}

Let's consider a specific example where the agent is in state $S$ and can take actions to move left, right, up, or down. The rewards for each action are as follows:
\begin{itemize}
    \item Moving left or right: $-1$
    \item Moving up or down: $-1$
\end{itemize}

The goal state (state $G$) has a reward of $+10$. Since the grid is deterministic, the agent moves to the desired state with probability $1$.

We can calculate the Q-values for each state-action pair using the Bellman equation and the given rewards. Let's start with the initial Q-values:
\[
\begin{aligned}
Q(S, \text{left}) &= 0 \\
Q(S, \text{right}) &= 0 \\
Q(S, \text{up}) &= 0 \\
Q(S, \text{down}) &= 0 \\
\end{aligned}
\]

To update the Q-values, we apply the Bellman equation for each state-action pair. For example, to update $Q(S, \text{left})$:
\[
\begin{aligned}
Q(S, \text{left}) &= -1 + 0.9 \times \max(Q(\text{next state}, \text{all actions})) \\
&= -1 + 0.9 \times \max(Q(S, \text{left}), Q(S, \text{right}), Q(S, \text{up}), Q(S, \text{down})) \\
&= -1 + 0.9 \times \max(0, 0, 0, 0) \\
&= -1
\end{aligned}
\]

Similarly, we can update $Q(S, \text{right})$, $Q(S, \text{up})$, and $Q(S, \text{down})$. After updating, the Q-values become:
\[
\begin{aligned}
Q(S, \text{left}) &= -1 \\
Q(S, \text{right}) &= -1 \\
Q(S, \text{up}) &= -1 \\
Q(S, \text{down}) &= -1 \\
\end{aligned}
\]

These updated Q-values reflect the expected cumulative rewards the agent can achieve from each state-action pair following an optimal policy in the grid world environment.

\end{document}
